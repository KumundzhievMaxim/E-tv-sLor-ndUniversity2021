\chapter{Medical Imaging}
\label{ch:rworks}

The imaging modalities used in biology and medicine are based on a variety of energy sources, including light, electrons, lasers, X-rays, radionuclides, ultrasound and nuclear magnetic resonance. The images produced span orders of magnitude in scale, ranging from molecules and cells to organ systems and the full body. The advantages and limitations of each imaging modality are primarily governed by the basic physical and biological principles which influence the way each energy form interacts with tissues, and by the specific engineering implementation for a particular medical or biological application.

\subsection{Modalities}
Each modality results from a different phenomenon of physics offers doctors (radiologists) an alternative view of the patient. Moreover each modality has it's risks, costs and benefits. All these factors have to be taken into account when choosing type of modality to be issued.

\subsubsection{X-Ray}
In 1895, Wilhelm Roentgen explored rays which could pass through wood and human tissues. He called them "x-rays" where "x" was considered as a place-holder for the unknown.

X-Rays are a form of electromagnetic radiation in the same spectrum as visible light and radio waves. Like light, x-rays can be considered as a energy of an x-ray photon. To make x-rays, it can be fired high-energy electrons into matter. In medical-imaging applications, x-rays are sent into tissue. 

Jointly x-rays interactions with matter it can be considered 3 common cases:
\begin{itemize}
    \item Photoelectric Effect
    \newline This effect is the principal effect that makes x-ray useful.
    \item Compton Interaction
    \newline Because of tissues tend to have little variations, Compton Interaction does not give precise information in terms of what is inside the body. 
    \item Rayleigh (Coherent) Scattering
    \newline Upon interacting with the attenuating medium, the photon does not have enough energy to liberate the electron from its bound state.
\end{itemize}

\subsubsection{Computed Tomography (CT)}
In simplified terms, the idea of computed tomography is to resolve a single slice of an object using many x-ray projections. As the gantry rotates, the scanner collects a 1D x-ray at each angle.

The CT scanners which designed for purpose of human diagnostic are generally capable of producing images with voxels, where voxel stand for representing a value on a regular grid in three-dimensional space    

The CT images which are span of x-rays are a form of ionizing radiation. Ionizing radiation is radiation with high enough energy that electrons can be ejected out of their orbitals, creating ions. These ions in large amounts can cause tissue and DNA damages. By this consideration medicine actively limits the amount of ionizing radiation the patient may potentially gets.

\item Contrast Agents
\newline
As the one of additional practices approaching CT for retrieving human body information, frequently substances can be introduced to the body to add the contrast, what is named as contrast agents. Often agents may be essentially useful for visualising of observations.          

\item Motion Artifacts
\newline
Usually it takes a few seconds to obtain one bit of CT data. The tidiness of the observations depends on whether the patient moved during the acquisition. If so, the resulting transformation will be inconsistent and the reconstructed image will contain errors. But, if the patient's motions are known, meaning a lot of artifacts can be corrected during the reconstruction. On top of it the few automatic methods can be applied to dare to sharpen the image by guessing the motions.

\subsubsection{PET}      





- How do we interpret images?
    - What is Image modalities
	- The types of Image modalities
    - Image modalities for spine
    - How complex problem is?
    
- The background problem definition
  
