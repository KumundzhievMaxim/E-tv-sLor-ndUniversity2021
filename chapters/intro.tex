Medical imaging consists of set of processes or techniques to create visual representations of the interior parts of the body such as organs or tissues for clinical purposes to monitor health, diagnose and treat diseases and injuries. Moreover, it also helps in creating database of anatomy and physiology.
A group of processes for creating visual representations of the interior parts of the body such as organs or tissues are called medical imaging. This technique is used for clinical purposes to monitor health, diagnose and treat diseases and injuries. Additionally, it also helps to generate a database of anatomy and physiology.

Owing to the advancements in the field today medical imaging has the ability to achieve information of human body for many useful clinical applications. Different types of medical imaging technology gives different information about the area of the body to be studied or medically treated.
There is a lot of progress in medical imaging today. Thereby this field enables reach information on the human body for many valuable clinical applications. Various types of medical imaging technology provide different information regarding the area of the body which needs to be studied or medically treated.
 
Organisations incorporating the medical imaging devices include freestanding radiology and pathology facilities as well as clinics and hospitals.
Freestanding radiology, pathology facilities, clinics, and hospitals are organizations where are medical imaging devices are used.

The use of medical imaging for diagnostic services is regarded as a significant confirmation of assessment and documentation of many diseases and ailments. High quality imaging improves medical decision making and can reduce unnecessary medical procedures.
Implementation of medical imaging in diagnostic services is considered to be an important validation of evaluation and documentation of many diseases and ailments. For improving medical decision-making and reducing pointless medical procedures high-quality imaging is used.

Earlier diagnosis included exploratory procedures to figure out issues of ageing person, children with chronic pain, detection of early diabetes and cancer. With the advent of medical imaging the vital information of health can be made available from time to time easily which can help diagnose illnesses like pneumonia, cancer, internal bleeding, brain injuries, and many more.

Earlier diagnosis contained exploratory procedures to find out issues of elderly people, children with chronic pain, detection of early cancer or diabetes. Illnesses like cancer, internal bleeding, brain injuries, pneumonia, and many more can be diagnosed much more easily with the advent of medical imaging because the vital information of health becomes more available.

Doctors perform medical imaging to determine the status of the organ and what treatments would be required for the recovery. The choice of imaging depends on the body being examined and the health concern of the patient. Therefore, patients are tested before if their body reacts affirmatively to the radiation used for medical imaging and making sure least possible amount of radiation is used for the process. Moreover, proper shielding is done to avoid other body parts from getting affected.

There are many different types of imaging, such as X-rays, CT (computed tomography) scans, MRI (magnetic resonance imaging) and ultrasound. Each imaging type uses a different technology to create an image. This increasing range of imaging types provides health professionals with many options for showing what is happening inside your body.

Medical imaging to capture the spine is a key tool used in clinical practice for diagnosis and treatment of patients suffering from spinal conditions. A common image modality used in clinics is Computed Tomography (CT), which provides a detailed view of the spinal anatomy. Once a scan of a patient is acquired the scan must be analyzed. 

One such analysis would be to locate and identify which vertebrae are visible within the scan. However, this is not a trivial task, is error-prone and can be time- consuming. First, as CT scans use X-rays the exposure to the patient is often limited by only scanning the part of the body which is of interest. This means that a limited number of vertebrae are captured in each scan. These scans, known as arbitrary Field-Of-View (FoV) scans, make the vertebrae harder to identify because a radiologist cannot simply count from the end of the column or because they miss important visual context. Second, the scans often include severe pathological cases which could mean that the spine is an abnormal shape or the scan could be post-op and contain metal implants causing imaging arti-facts. Third, many vertebrae have similar appearance and are hard to tell apart without contextual, anatomical information. 
An algorithm which could perform the task of vertebrae localization and identification could not only inform radiologists and speed up their workflow, it could also be used as prior information for subsequent algorithms, for example, a lesion or fracture detection algorithm. In addition, automation could be used to perform a survey over a large dataset of spines to record information about a population, a task which is difficult to do manually by human experts.
The same reasons that make this problem hard for radiologists also make it difficult for an algorithm. In 2012, Glocker et al. [4]. proposed a solution to the problem which regressed the centroid positions however the method made assumptions about the shape of the spine and thus did not work well on patho- logical cases. Later, in [5] this was improved upon from their previous solution by turning the problem from a regression into a dense classification problem. This was done by generating a dense labelling, which is a label given to each pixel in the scan, from the ground-truth centroid annotations. The dense labelling is then converted back to the sparse labelling of centroid positions at the end of the processing pipeline. This approach worked much better for pathological cases. Both approaches [4,5] used random forests but there has since been an emergence of convolutional neural networks (CNN) in medical image analysis [3,7,10] and better results have been achieved for the vertebrae localization task with deep learning [2]. Some use a U-net architecture, such as Yang et al [11], which is a popular architecture for image segmentation problems [9]. In 2018, Liao et al. [6] published a state-of-the-art solution which regresses the positions of the centroids using a CNN combined with a Recurrent Neural Network (RNN) to capture the ordering of the vertebrae and to incorporate long-range contex- tual information. Based on the literature, we decided to develop a new approach which should have the following features: 1) Use CNNs, 2) Capture the order- ing of the vertebrae to improve accuracy, 3) Use a Dense Labelling strategy and turn the problem into a classification task, 4) Capture short and long-range contextual information from the scan.
