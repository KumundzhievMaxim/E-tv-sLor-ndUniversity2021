\chapter{Introduction}
\label{ch:introduction}


The main part of the human axial skeleton consists of 33-34 vertebrae which coherently connected to each other in the upright position \cite{Ahlberg2005}. Types of vertebraes can be considered as cervical, thoracic, lumbar, sacral (fused into a sacrum) and coccyx. Vertebraes which mostly constitute human's back, performs a vital functions of backing, protection of spinal cord and are involved in the activities of torso and head.

Since, spine functionalities come off as the fundamental movement-wise operations, the science has always been keenly paying attention to the survey of more rapid, substantive and accurate solutions for spine analysis.
 
Whenever the speech is about the analysis of medical images we always refer to work with the images themselves, clearly meaning analysing assay of certain set of numbers. Typically, the most challenging problems in analyses are classification of objects, as well as segmentation of different parts of human body in various image modalities.  
 
The advent and promotion of AI, along with growing of computing power, \cite{Pham2000} grants a huge impetus for medical research field. The rise of accessibility of different medical data sets has significantly increased including the surge of sundry vertebrae research papers. Nowadays, scientists and researchers make use of AI to automatically recognize complex patterns in medical data and as a result provide quantitative assessments of different characteristics.  
 
Jointly automatic vertebrae segmentation, precise analysis of the spinal structures is an essential condition in all the clinical applications of spinal analysis. Knowledge of the detailed shape of individual vertebrae can considerably aid early diagnosis, surgical planning and follow-up assessment of a number of spinal pathologies, such as degenerative disorders, spinal deformities, trauma and tumors. Classification and segmentation of fractured vertebrae by computer-assisted techniques may therefore provide additional support to diagnosis and treatment of vertebral fractures. 

In the thesis work I proposed two-stage approach for the automatic vertebrae segmentation problem. Whereas the first stage is so called detection and the second stage localization. 
Detection part reveals the occurrence of vertebrae on the input CT images. Following that identification part recognises the concrete vertebrae within corresponding region-of-interest.